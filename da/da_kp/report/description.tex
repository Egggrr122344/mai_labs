\section*{Курсовая работа: Реализация алгоритма LZ77}

\subsection*{Описание задания}

В рамках данной курсовой работы требуется реализовать алгоритм LZ77 для сжатия и восстановления текста. Работа должна учитывать следующие условия:

\begin{enumerate}
  \item \textbf{Алгоритм сжатия:}
  \begin{itemize}
    \item Поиск совпадений выполняется по всему ранее просмотренному тексту (без ограничения окна).
    \item На вход подается команда \texttt{compress} с текстом, состоящим только из строчных латинских букв.
    \item Результатом работы является последовательность троек вида $(offset, length, char)$, где:
    \begin{itemize}
      \item $offset$ --- смещение от текущей позиции к началу совпадения.
      \item $length$ --- длина совпадения.
      \item $char$ --- символ, следующий за совпадением. Если его нет, то тройка заканчивается на $offset$ и $length$.
    \end{itemize}
  \end{itemize}

  \item \textbf{Алгоритм восстановления:}
  \begin{itemize}
    \item На вход подается команда \texttt{decompress} с последовательностью троек вида $(offset, length, char)$.
    \item Результатом работы должен быть текст, соответствующий сжатой последовательности троек.
    \item Если символ $char$ отсутствует в тройке, это означает конец текста.
  \end{itemize}

  \item \textbf{Формат ввода и вывода:}
  \begin{itemize}
    \item \textbf{Сжатие:}
      \begin{itemize}
        \item На вход: \texttt{compress <text>}.
        \item На выход: последовательность троек, каждая из которых выводится на отдельной строке.
      \end{itemize}
    \item \textbf{Восстановление:}
      \begin{itemize}
        \item На вход: \texttt{decompress <triplets>}.
        \item На выход: восстановленный текст.
      \end{itemize}
  \end{itemize}
\end{enumerate}

\subsection*{Описание алгоритма}

\subsubsection*{Алгоритм сжатия (LZ77 Compress)}

Алгоритм сжатия LZ77 работает на основе нахождения повторяющихся последовательностей символов в тексте. Основные шаги:

\begin{enumerate}
  \item \textbf{Инициализация:}
    \begin{itemize}
      \item Установите текущую позицию в начало текста.
    \end{itemize}

  \item \textbf{Поиск совпадений:}
    \begin{itemize}
      \item Для текущей позиции выполните поиск наибольшего совпадения с ранее просмотренным текстом.
      \item Найденное совпадение характеризуется двумя параметрами:
        \begin{itemize}
          \item $offset$ --- расстояние от текущей позиции до начала совпадения.
          \item $length$ --- длина совпадающей последовательности.
        \end{itemize}
    \end{itemize}

  \item \textbf{Добавление новой тройки:}
    \begin{itemize}
      \item Если совпадение найдено, то за ним следует символ $char$, который не входит в совпадение.
      \item Если совпадения нет, создайте тройку $(0, 0, char)$, где $char$ --- текущий символ текста.
    \end{itemize}

  \item \textbf{Обновление текущей позиции:}
    \begin{itemize}
      \item Переместите текущую позицию вперед на длину совпадения плюс один символ.
    \end{itemize}

  \item Повторите шаги 2--4, пока не будет обработан весь текст.
\end{enumerate}

\subsubsection*{Алгоритм восстановления (LZ77 Decompress)}

Для восстановления текста из троек $(offset, length, char)$ используйте следующие шаги:

\begin{enumerate}
  \item \textbf{Инициализация:}
    \begin{itemize}
      \item Создайте пустую строку для результата.
    \end{itemize}

  \item \textbf{Обработка троек:}
    \begin{itemize}
      \item Для каждой тройки $(offset, length, char)$ выполните:
        \begin{itemize}
          \item Если $length > 0$, скопируйте $length$ символов, начиная с позиции $len(result) - offset$ в уже восстановленном тексте.
          \item Если $char$ присутствует, добавьте его в конец текста.
        \end{itemize}
    \end{itemize}

  \item Повторяйте шаг 2 для всех троек, пока они не будут обработаны.
\end{enumerate}

\subsection*{Пример пошагового выполнения}

\subsubsection*{Входной текст: \texttt{abracadabra}}

\begin{enumerate}
  \item Начинаем с позиции 0:
    \begin{itemize}
      \item Нет совпадений, добавляем $(0, 0, a)$.
      \item Результат: \texttt{a}.
    \end{itemize}

  \item Позиция 1:
    \begin{itemize}
      \item Нет совпадений, добавляем $(0, 0, b)$.
      \item Результат: \texttt{ab}.
    \end{itemize}

  \item Позиция 2:
    \begin{itemize}
      \item Нет совпадений, добавляем $(0, 0, r)$.
      \item Результат: \texttt{abr}.
    \end{itemize}

  \item Позиция 3:
    \begin{itemize}
      \item Совпадение длиной 1 (\texttt{a}), добавляем $(3, 1, c)$.
      \item Результат: \texttt{abrac}.
    \end{itemize}

  \item Продолжайте до конца текста.
\end{enumerate}

\subsection*{Примеры тестов}

\subsubsection*{Тест 1: Сжатие}

\begin{verbatim}
Ввод:
compress abracadabra

Вывод:
0 0 a
0 0 b
0 0 r
3 1 c
2 1 d
7 4
\end{verbatim}

\subsubsection*{Тест 2: Восстановление}

\begin{verbatim}
Ввод:
decompress
0 0 a
0 0 b
0 0 r
3 1 c
2 1 d
7 4

Вывод:
abracadabra
\end{verbatim}


\section*{Выводы}

\subsection*{Преимущества алгоритма LZ77}
\begin{itemize}
    \item \textbf{Высокая степень сжатия}: Алгоритм эффективно уменьшает размер данных за счёт использования повторяющихся последовательностей.
    \item \textbf{Простота реализации}: Основной механизм поиска совпадений и кодирования легко реализовать с помощью базовых структур данных.
    \item \textbf{Отсутствие необходимости в словаре}: LZ77 не требует заранее определённого словаря, поскольку использует уже просмотренную часть текста.
    \item \textbf{Гибкость}: Подходит для различных типов данных, включая текстовые и бинарные файлы.
\end{itemize}

\subsection*{Недостатки алгоритма LZ77}
\begin{itemize}
    \item \textbf{Высокая вычислительная сложность}: Поиск совпадений требует значительных временных затрат, особенно при обработке длинных входных строк.
    \item \textbf{Ограничения для больших данных}: Производительность может ухудшаться при увеличении объёма данных из-за необходимости анализа всей ранее просмотренной части текста.
    \item \textbf{Неоптимальное сжатие для коротких строк}: На малых данных алгоритм может быть менее эффективным, так как число повторяющихся последовательностей ограничено.
    \item \textbf{Проблемы с памятью}: При сжатии больших строк увеличивается потребность в памяти для хранения просмотренной части текста.
\end{itemize}

\subsection*{Возможности улучшения алгоритма}
\begin{itemize}
    \item \textbf{Использование скользящего окна}: Ограничение размера области поиска (например, последние $N$ символов) для уменьшения сложности и улучшения производительности.
    \item \textbf{Оптимизация структуры поиска}: Применение хеш-таблиц, деревьев суффиксов или других эффективных структур данных для ускорения поиска совпадений.
    \item \textbf{Параллелизация}: Разделение входных данных на блоки и выполнение сжатия параллельно для увеличения скорости обработки.
    \item \textbf{Адаптивные параметры}: Динамическая настройка параметров, таких как размер окна и минимальная длина совпадения, для достижения баланса между степенью сжатия и скоростью.
    \item \textbf{Сжатие с использованием статистики}: Интеграция с энтропийным кодированием, например, методом Хаффмана или арифметическим кодированием, для дальнейшего уменьшения размера выходных данных.
\end{itemize}

\subsection*{Заключение}
Алгоритм LZ77 демонстрирует высокую эффективность при сжатии данных за счёт поиска повторяющихся последовательностей. Однако его вычислительная сложность остаётся значительным препятствием для обработки больших данных. При внедрении описанных выше методов можно значительно улучшить качество алгоритма и ускорить временную сложность.
В результате работы над курсовой работой я изучил материалы, относящиеся к алгоритмам сжатия, выявил различия и особенности алгоритма LZ77, реализовал его на практике и протестировал на длинных строках вплоть до 500000 символов.
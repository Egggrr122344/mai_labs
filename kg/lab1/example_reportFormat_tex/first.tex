\section*{Тема лабораторной работы: Основы 3D-графики и проекция}

\subsection{Цель лабораторной работы}

В этой лабораторной работе вы познакомитесь с основами 3D-графики: построением
простых 3D-объектов, проекцией на 2D-плоскость, а также научитесь работать с матрицами
перспективы и ортографической проекции

\subsection{Требования}

Вы должны использовать С++ (OpenGL+SFML).
Программа должна работать в реальном времени,с возможностью динамической смены
проекции и трансформаций объектов.
Все объекты должны корректно отрисовываться с учетом проекции и иметь возможность
взаимодействия с пользователем.

\subsection{Вариант 2: Построение пирамиды с перспективной проекцией}

Постройте 3D-пирамиду (с квадратным основанием).
 Примените перспективную проекцию для отображения пирамиды.
 Реализуйте вращение пирамиды вокруг всех осей с помощью клавиш управления.
 Дополнительно: Добавьте динамическое изменение угла обзора (field of view) и наблюдайте,
 как это влияет на проекцию
 \pagebreak